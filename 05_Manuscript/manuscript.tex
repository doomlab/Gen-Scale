% Options for packages loaded elsewhere
\PassOptionsToPackage{unicode}{hyperref}
\PassOptionsToPackage{hyphens}{url}
%
\documentclass[
  man]{apa6}
\usepackage{amsmath,amssymb}
\usepackage{iftex}
\ifPDFTeX
  \usepackage[T1]{fontenc}
  \usepackage[utf8]{inputenc}
  \usepackage{textcomp} % provide euro and other symbols
\else % if luatex or xetex
  \usepackage{unicode-math} % this also loads fontspec
  \defaultfontfeatures{Scale=MatchLowercase}
  \defaultfontfeatures[\rmfamily]{Ligatures=TeX,Scale=1}
\fi
\usepackage{lmodern}
\ifPDFTeX\else
  % xetex/luatex font selection
\fi
% Use upquote if available, for straight quotes in verbatim environments
\IfFileExists{upquote.sty}{\usepackage{upquote}}{}
\IfFileExists{microtype.sty}{% use microtype if available
  \usepackage[]{microtype}
  \UseMicrotypeSet[protrusion]{basicmath} % disable protrusion for tt fonts
}{}
\makeatletter
\@ifundefined{KOMAClassName}{% if non-KOMA class
  \IfFileExists{parskip.sty}{%
    \usepackage{parskip}
  }{% else
    \setlength{\parindent}{0pt}
    \setlength{\parskip}{6pt plus 2pt minus 1pt}}
}{% if KOMA class
  \KOMAoptions{parskip=half}}
\makeatother
\usepackage{xcolor}
\usepackage{graphicx}
\makeatletter
\newsavebox\pandoc@box
\newcommand*\pandocbounded[1]{% scales image to fit in text height/width
  \sbox\pandoc@box{#1}%
  \Gscale@div\@tempa{\textheight}{\dimexpr\ht\pandoc@box+\dp\pandoc@box\relax}%
  \Gscale@div\@tempb{\linewidth}{\wd\pandoc@box}%
  \ifdim\@tempb\p@<\@tempa\p@\let\@tempa\@tempb\fi% select the smaller of both
  \ifdim\@tempa\p@<\p@\scalebox{\@tempa}{\usebox\pandoc@box}%
  \else\usebox{\pandoc@box}%
  \fi%
}
% Set default figure placement to htbp
\def\fps@figure{htbp}
\makeatother
\setlength{\emergencystretch}{3em} % prevent overfull lines
\providecommand{\tightlist}{%
  \setlength{\itemsep}{0pt}\setlength{\parskip}{0pt}}
\setcounter{secnumdepth}{-\maxdimen} % remove section numbering
% Make \paragraph and \subparagraph free-standing
\makeatletter
\ifx\paragraph\undefined\else
  \let\oldparagraph\paragraph
  \renewcommand{\paragraph}{
    \@ifstar
      \xxxParagraphStar
      \xxxParagraphNoStar
  }
  \newcommand{\xxxParagraphStar}[1]{\oldparagraph*{#1}\mbox{}}
  \newcommand{\xxxParagraphNoStar}[1]{\oldparagraph{#1}\mbox{}}
\fi
\ifx\subparagraph\undefined\else
  \let\oldsubparagraph\subparagraph
  \renewcommand{\subparagraph}{
    \@ifstar
      \xxxSubParagraphStar
      \xxxSubParagraphNoStar
  }
  \newcommand{\xxxSubParagraphStar}[1]{\oldsubparagraph*{#1}\mbox{}}
  \newcommand{\xxxSubParagraphNoStar}[1]{\oldsubparagraph{#1}\mbox{}}
\fi
\makeatother
% definitions for citeproc citations
\NewDocumentCommand\citeproctext{}{}
\NewDocumentCommand\citeproc{mm}{%
  \begingroup\def\citeproctext{#2}\cite{#1}\endgroup}
\makeatletter
 % allow citations to break across lines
 \let\@cite@ofmt\@firstofone
 % avoid brackets around text for \cite:
 \def\@biblabel#1{}
 \def\@cite#1#2{{#1\if@tempswa , #2\fi}}
\makeatother
\newlength{\cslhangindent}
\setlength{\cslhangindent}{1.5em}
\newlength{\csllabelwidth}
\setlength{\csllabelwidth}{3em}
\newenvironment{CSLReferences}[2] % #1 hanging-indent, #2 entry-spacing
 {\begin{list}{}{%
  \setlength{\itemindent}{0pt}
  \setlength{\leftmargin}{0pt}
  \setlength{\parsep}{0pt}
  % turn on hanging indent if param 1 is 1
  \ifodd #1
   \setlength{\leftmargin}{\cslhangindent}
   \setlength{\itemindent}{-1\cslhangindent}
  \fi
  % set entry spacing
  \setlength{\itemsep}{#2\baselineskip}}}
 {\end{list}}
\usepackage{calc}
\newcommand{\CSLBlock}[1]{\hfill\break\parbox[t]{\linewidth}{\strut\ignorespaces#1\strut}}
\newcommand{\CSLLeftMargin}[1]{\parbox[t]{\csllabelwidth}{\strut#1\strut}}
\newcommand{\CSLRightInline}[1]{\parbox[t]{\linewidth - \csllabelwidth}{\strut#1\strut}}
\newcommand{\CSLIndent}[1]{\hspace{\cslhangindent}#1}
\ifLuaTeX
\usepackage[bidi=basic]{babel}
\else
\usepackage[bidi=default]{babel}
\fi
\babelprovide[main,import]{english}
% get rid of language-specific shorthands (see #6817):
\let\LanguageShortHands\languageshorthands
\def\languageshorthands#1{}
\ifLuaTeX
  \usepackage[english]{selnolig} % disable illegal ligatures
\fi
% Manuscript styling
\usepackage{upgreek}
\captionsetup{font=singlespacing,justification=justified}

% Table formatting
\usepackage{longtable}
\usepackage{lscape}
% \usepackage[counterclockwise]{rotating}   % Landscape page setup for large tables
\usepackage{multirow}		% Table styling
\usepackage{tabularx}		% Control Column width
\usepackage[flushleft]{threeparttable}	% Allows for three part tables with a specified notes section
\usepackage{threeparttablex}            % Lets threeparttable work with longtable

% Create new environments so endfloat can handle them
% \newenvironment{ltable}
%   {\begin{landscape}\centering\begin{threeparttable}}
%   {\end{threeparttable}\end{landscape}}
\newenvironment{lltable}{\begin{landscape}\centering\begin{ThreePartTable}}{\end{ThreePartTable}\end{landscape}}

% Enables adjusting longtable caption width to table width
% Solution found at http://golatex.de/longtable-mit-caption-so-breit-wie-die-tabelle-t15767.html
\makeatletter
\newcommand\LastLTentrywidth{1em}
\newlength\longtablewidth
\setlength{\longtablewidth}{1in}
\newcommand{\getlongtablewidth}{\begingroup \ifcsname LT@\roman{LT@tables}\endcsname \global\longtablewidth=0pt \renewcommand{\LT@entry}[2]{\global\advance\longtablewidth by ##2\relax\gdef\LastLTentrywidth{##2}}\@nameuse{LT@\roman{LT@tables}} \fi \endgroup}

% \setlength{\parindent}{0.5in}
% \setlength{\parskip}{0pt plus 0pt minus 0pt}

% Overwrite redefinition of paragraph and subparagraph by the default LaTeX template
% See https://github.com/crsh/papaja/issues/292
\makeatletter
\renewcommand{\paragraph}{\@startsection{paragraph}{4}{\parindent}%
  {0\baselineskip \@plus 0.2ex \@minus 0.2ex}%
  {-1em}%
  {\normalfont\normalsize\bfseries\itshape\typesectitle}}

\renewcommand{\subparagraph}[1]{\@startsection{subparagraph}{5}{1em}%
  {0\baselineskip \@plus 0.2ex \@minus 0.2ex}%
  {-\z@\relax}%
  {\normalfont\normalsize\itshape\hspace{\parindent}{#1}\textit{\addperi}}{\relax}}
\makeatother

\makeatletter
\usepackage{etoolbox}
\patchcmd{\maketitle}
  {\section{\normalfont\normalsize\abstractname}}
  {\section*{\normalfont\normalsize\abstractname}}
  {}{\typeout{Failed to patch abstract.}}
\patchcmd{\maketitle}
  {\section{\protect\normalfont{\@title}}}
  {\section*{\protect\normalfont{\@title}}}
  {}{\typeout{Failed to patch title.}}
\makeatother

\usepackage{xpatch}
\makeatletter
\xapptocmd\appendix
  {\xapptocmd\section
    {\addcontentsline{toc}{section}{\appendixname\ifoneappendix\else~\theappendix\fi\\: #1}}
    {}{\InnerPatchFailed}%
  }
{}{\PatchFailed}
\keywords{keywords\newline\indent Word count: X}
\DeclareDelayedFloatFlavor{ThreePartTable}{table}
\DeclareDelayedFloatFlavor{lltable}{table}
\DeclareDelayedFloatFlavor*{longtable}{table}
\makeatletter
\renewcommand{\efloat@iwrite}[1]{\immediate\expandafter\protected@write\csname efloat@post#1\endcsname{}}
\makeatother
\usepackage{lineno}

\linenumbers
\usepackage{csquotes}
\usepackage{bookmark}
\IfFileExists{xurl.sty}{\usepackage{xurl}}{} % add URL line breaks if available
\urlstyle{same}
\hypersetup{
  pdftitle={The title},
  pdfauthor={First Author1 \& Ernst-August Doelle1,2},
  pdflang={en-EN},
  pdfkeywords={keywords},
  hidelinks,
  pdfcreator={LaTeX via pandoc}}

\title{The title}
\author{First Author\textsuperscript{1} \& Ernst-August Doelle\textsuperscript{1,2}}
\date{}


\shorttitle{Title}

\authornote{

Add complete departmental affiliations for each author here. Each new line herein must be indented, like this line.
Enter author note here.

The authors made the following contributions. First Author: Conceptualization, Writing - Original Draft Preparation, Writing - Review \& Editing; Ernst-August Doelle: Writing - Review \& Editing, Supervision.

Correspondence concerning this article should be addressed to First Author, Postal address. E-mail: \href{mailto:my@email.com}{\nolinkurl{my@email.com}}

}

\affiliation{\vspace{0.5cm}\textsuperscript{1} Wilhelm-Wundt-University\\\textsuperscript{2} Konstanz Business School}

\abstract{%
One or two sentences providing a \textbf{basic introduction} to the field, comprehensible to a scientist in any discipline.
Two to three sentences of \textbf{more detailed background}, comprehensible to scientists in related disciplines.
One sentence clearly stating the \textbf{general problem} being addressed by this particular study.
One sentence summarizing the main result (with the words ``\textbf{here we show}'' or their equivalent).
Two or three sentences explaining what the \textbf{main result} reveals in direct comparison to what was thought to be the case previously, or how the main result adds to previous knowledge.
One or two sentences to put the results into a more \textbf{general context}.
Two or three sentences to provide a \textbf{broader perspective}, readily comprehensible to a scientist in any discipline.
}



\begin{document}
\maketitle

Death is something each of us must learn to cope with, whether in
healthy ways or less so. These issues may be at front of mind for many
in light of the COVID-19 pandemic. Various existential philosophers and
psychologists have proposed ways in which we deal with the awareness of
death and the anxiety this awareness often causes. Psychoanalyst Erik
Erikson (1950) proposed that during mid-life one becomes acutely aware
of their oncoming death and is motivated to care for things which will
outlast themselves. He called this act of caring generativity. In \emph{The
Denial of Death}, philosopher Ernest Becker (1973) posits that humans
undertake immortality projects to curb their sense of vulnerability to
death. Similarly, psychiatrist Robert Jay Lifton (1979), a mentee of
Erikson, described the awareness of death as being ever present and
motivating us to create symbols, thereby allowing us to imagine
ourselves as symbolically immortalized. Existential psychiatrist Irvin
Yalom (2008) notes that many of his clients experiencing anxiety about
their death take comfort in ``rippling,'' the idea that one's lasting
effects on the world will ripple out and influence the world after they
have died.

Although these thinkers use different terminology, there are several
common themes among their ideas. (1) Our physical death is an
inevitability, and we often find our awareness of its inevitability to
be aversive. This aversion may be referred to variously as angst,
death-anxiety, despair, being-towards-death, terror, and so on. However,
(2) we take comfort in the idea that other, non-physical parts of us
continue to exist indefinitely after our biological death, through
mechanisms such as the heroic archetype and symbolic self. (3) Finally,
we can take action to promote these non-physical aspects of the self,
such as through search for meaning, sense of immortality, care,
generativity, and rippling.

One of these bodies of thought, called symbolic immortality, was
originally theorized by Lifton (1979), who thought that awareness of
death drives a fundamental human desire for a sense of continuity
lasting beyond the lifespan. Essentially, humans are meaning-seeking
creatures, and throughout our lives, this search for meaning involves an
evolving psychological imagery of life and death. Death, or the
transient nature of life, threatens our search for meaning. Lifton
thought that if we could achieve what we believe to be some form of
immortality, we could overcome this loss of meaning, and the awareness
of death could instead drive an inner vitality (imagery associated with
connection, integrity, and movement). If this drive toward vitality is
lost, we are vulnerable to a psychic numbness or death-in-life (imagery
associated with separation, disintegration, and stasis). In Lifton's own
words, ``Death does indeed bring about biological and psychic
annihilation. But life includes symbolic perceptions of connections that
precede and outlast that annihilation'' (1979, p.~18).

Lifton (1979) proposed five modes of experience or ways of achieving
symbolic immortality: The biological (or biosocial) mode in which one
lives on through their genetic and sociocultural progeny, the creative
mode in which one's accomplishments and contribution outlast oneself,
the natural mode in which one feels they are a part of the broader
universe, the spiritual mode in which one seeks to transcend the
physical realm to a higher spiritual realm beyond death, and the mode of
experiential transcendence in which one experiences a phenomenological
state of flow. The experiential mode must occur in the context of at
least one of the other four to really be considered transcendent, but it
is thought to have a great capacity to bring about personal change.

Claims of how we suppress death-anxiety have been investigated
experimentally, primarily through the paradigm of Terror Management
Theory (TMT). Based on the theories of Ernest Becker, TMT posits that
human awareness of death is always present to some degree. This
awareness of our inevitable death, coupled with a strong aversion to
thoughts of death, causes terror and is pushed out of our consciousness
by our creation of meaning systems (Greenberg, Pyszczynski, \& Solomon, 1986). TMT proposes that
self-esteem, interpersonal relationships, and cultural worldview work
together to buffer against our anxiety about death. It is assumed that
these buffers suppress thoughts of death by providing a sense of
symbolic immortality, though little systematic research has been
conducted on this construct. The results of this buffering process are
not always positive. For example, experimentally priming mortality
salience can lead to more positive attitudes toward in-group members but
harsher negative attitudes toward out-group members (Greenberg et al., 1990).

TMT refers to a person's awareness of death as mortality salience (MS).
The MS hypothesis of TMT posits than an increase in one's awareness of
death causes an increase in compensatory behaviors to lower their death
anxiety, either by distracting from the awareness of death or by the
promotion of meaningful cultural worldviews. In the MS paradigm,
experimentally priming a participant's awareness of death (for example,
by having participants write about death and then complete a distraction
task) is thought to cause an increase in compensatory buffers. A
meta-analysis of 277 experiments found mortality salience to have a
robust, moderate overall effect size: \emph{r}(276) = 0.35, \emph{p} = .00
(Burke, Martens, \& Faucher, 2010). Altogether, these experiments provide convincing evidence
for TMT and the MS hypothesis in particular.

Though some avoidance of (or buffering against) death anxiety is thought
to be universal and has the potential to increase interpersonal
conflict, awareness of death through symbolic immortality may also have
potential as a positive force. In particular, it is thought to be an
underlying motive for what Erikson referred to as generativity.
Generativity is the seventh of eight proposed stages in Erikson's (1950)
theory of psychosocial development, which he associated with midlife and
described as ``the concern in establishing and guiding the next
generation'' (Erikson, 1963, p.~267). Little systematic research was
conducted on this subject until the 1980's. Kotre (1984) expanded on the
theory and proposed that the drive for generativity was related to a
motive to expand the sense of self beyond the lifetime, especially in
light of the fear of death.

McAdams and de St.~Aubin (1992) sought to formalize the study
of generativity as a multidimensional construct. Their seven components
of generativity include cultural demand, inner desire (for symbolic
immortality and community), concern (for the next generation), belief
(in the human species), commitment, action, and narration (of
generativity within one's life story). In addition to a quantitative
measure of generative concern (the Loyola Generativity Scale), they
developed a system for content analysis of autobiographical episodes
pertaining to generativity, and symbolic immortality is one of the five
themes they found. Here they define symbolic immortality as ``any
reference to leaving a legacy, having an enduring influence, or leaving
behind products that will outlive one's physical existence,'' a theme
clearly related to both Lifton's and Erikson's theories (Aust et al., 2022; McAdams \& St. Aubin, 1992, p. 1011).

(Aust et al., 2022; Mathews \& Kling, 1988)

(McAdams \& de St.~Aubin, 1992, p.~1011).

These research areas depend on the construct of symbolic immortality for
their theoretical frameworks, but few researchers have attempted to
systematically and quantitatively assess this construct. Two attempts
have been made to develop such a measurement: Drolet's (1990) Sense of
Symbolic Immortality Scale and Mathews and Kling (1988) measure of
symbolic immortality, based on an original questionnaire by Mathews and
Mister (1987).

Drolet (1990) developed the Sense of Symbolic Immortality Scale based on
Robert J. Lifton's theory of symbolic immortality and its five modes of
experience. Drolet studied 136 adults, ages 18-30 and 30-40, and
hypothesized that those in their 30's (established adults) would have a
greater sense of symbolic immortality than the young adults (18-30). The
measure is inherently subjective, not only by the nature of self-report,
but in that the scale seeks to measure what a person \emph{believes} and how
they \emph{feel} about these subjects. The scale as a whole had a high
internal consistency (\emph{\$\textbackslash alpha\$} = .91) and test-retest reliability
was \emph{r} = .97. Internal consistency of subscales for the five
theoretical modes of immortality was mixed. Of the five, spiritual
immortality was the most distinct from the scale as a whole and the
other subscales. Factor analysis showed three factors, mapping onto
biosocial, creative, and spiritual. The transcendent and natural items
may be closely related to biosocial.

Moving beyond the scale development itself (still Drolet 1990), SSI
correlated negatively with death anxiety (Templer's Death Anxiety Scale)
and had a strong (\emph{r} = .84) positive relationship with purpose in life
(Maholick's Purpose in Life Test). In interpreting the very strong
correlation, the author suggests that SSI is a broader construct than
Purpose in Life and the scale itself may be less prone to social
desirability effects than the PIL, although this had not been directly
tested. Age group was also related, with established adults having a
higher SSI, particularly in the biosocial and creative domains.

We see multiple issues with using the Symbolic Immortality Scale. First,
the study was underpowered, conducting exploratory factor analysis of 67
items using a sample of 136. Second, the scale was developed in French,
and we do not take for granted the psychometric properties of a
translated version. Third and most fundamentally, the scale has poor
face validity and appears to measure the constructs theorized to
symbolically immortalize rather than a sense of symbolic immortality
directly. For example, the scale includes items such as ``My sex life
contributes greatly to my well-being'', ``Intimate relationships scare
me'', and ``I am sure of who I am.'' Although related to the constructs
(such as interpersonal relationships and self-esteem) which
theoretically help cope with death, it is unclear how these items
represent the construct of symbolic immortality itself.

Mathews and Mister (1987) also developed a scale pertaining to symbolic
immortality, sensation seeking, and psychic numbness in a study
including 400 adults. Experiential transcendence was operationalized as
something like Zuckerman's (1979) sensation seeking, which may not fully
capture the original intent (the experience of losing oneself). Items
were mapped onto five factors, and the five factors largely aligned with
Lifton's constructs. Although internal consistency was at least
acceptable for each factor, goodness of fit statistics are not reported.
Some studies have used a revised version of the scale by Mathews and
Kling (1988), who adapted it for a study on prosocial behavior in the
context of nonprofit volunteer motivation. They reported similar results
for their revised scale. The items on these scales seem to have more
face validity than the scale by Drolet, but some factors seem more
behavioral and unnecessarily specific: pertaining to one's religiosity
or biological children, whereas Lifton's theory allows for a broader
interpretation of these dimensions. The Nature and Creative factors seem
most useful and theoretically aligned with Lifton.

Much more advanced factor analysis methods have been developed since the
1980s, but to our knowledge, these scales have not been tested with more
robust tools. The goal of the present research is to develop an
up-to-date symbolic immortality scale that more directly measures one's
sense of symbolic immortality and which contains items more generally
applicable to broad groups of participants (e.g., regardless of a
person's religious beliefs and parental status). We have attempted to
use current best practices for scale development and analysis.

\section{Method}\label{method}

\subsection{Participants}\label{participants}

\subsection{Material}\label{material}

\subsection{Procedure}\label{procedure}

\subsection{Data analysis}\label{data-analysis}

\section{Results}\label{results}

\subsection{Data Screening}\label{data-screening}

\begin{itemize}
\tightlist
\item
  try running with outliers as well --\textgreater{} if differences: then talk
  about what is going on with the outlier people
\end{itemize}

\section{Assumptions -------------------------------------------------------------}\label{assumptions--}

\section{Linearity}\label{linearity}

random = rchisq(nrow(noout{[} , 37:97{]}), 7) fake = lm(random\textasciitilde.,
data=noout{[} , 37:97{]})

standardized = rstudent(fake) \{qqnorm(standardized) abline(0,1)\} \#seems
okay

\section{Normality}\label{normality}

hist(standardized, breaks=15) \#a little skewed but mostly centered on 0
and between -2 and 2 so okay

\section{Homog/s}\label{homogs}

fitvalues = scale(fake\$fitted.values) \{plot(fitvalues, standardized)
abline(0,0) abline(v = 0)\} \#some homogeneity issues but we'll accept it

write.csv(noout, file = ``sona\_data\_screened.csv'', row.names = F)
library(beepr) beep(sound = 5)

```

\section{Discussion}\label{discussion}

\newpage

\section{References}\label{references}

\phantomsection\label{refs}
\begin{CSLReferences}{1}{0}
\bibitem[\citeproctext]{ref-aust2022}
Aust, F., Barth, M., Diedenhofen, B., Stahl, C., Casillas, J. V., \& Siegel, R. (2022). \emph{Papaja: Prepare american psychological association journal articles with r markdown}. Retrieved from \url{https://CRAN.R-project.org/package=papaja}

\bibitem[\citeproctext]{ref-burke2010}
Burke, B. L., Martens, A., \& Faucher, E. H. (2010). Two Decades of Terror Management Theory: A Meta-Analysis of Mortality Salience Research. \emph{Personality and Social Psychology Review}, \emph{14}(2), 155--195. \url{https://doi.org/10.1177/1088868309352321}

\bibitem[\citeproctext]{ref-greenberg1986}
Greenberg, J., Pyszczynski, T., \& Solomon, S. (1986). \emph{The Causes and Consequences of a Need for Self-Esteem: A Terror Management Theory}. New York: Springer-Verlag.

\bibitem[\citeproctext]{ref-greenberg1990}
Greenberg, J., Solomon, S., Veeder, M., Pyszczynski, T., Rosenblatt, A., Kirkland, S., \& Lyon, D. (1990). Evidence for Terror Management Theory II: The Effects of Mortality Salience on Reactions to Those Who Threaten or Bolster the Cultural Worldview. \emph{Journal of Personality and Social Psychology}, \emph{58}(2), 308--318.

\bibitem[\citeproctext]{ref-mathews1988}
Mathews, R. C., \& Kling, K. J. (1988). Self-Transcendence, Time Perspective, and Prosocial Behavior. \emph{Journal of Voluntary Action Research}, \emph{17}(2), 4--24. \url{https://doi.org/10.1177/089976408801700202}

\bibitem[\citeproctext]{ref-mcadams1992}
McAdams, D. P., \& St. Aubin, E. de. (1992). A theory of generativity and its assessment through self-report, behavioral acts, and narrative themes in autobiography. \emph{Journal of Personality and Social Psychology}, \emph{62}(6), 1003--1015.

\end{CSLReferences}


\end{document}
